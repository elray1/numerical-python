\documentclass[]{article}

\usepackage{amsmath}
\usepackage[margin=1.5cm]{geometry}

%opening
\title{Example problem Ch 5}
\author{}

\begin{document}

\maketitle

\begin{abstract}

\end{abstract}

\section{Australian Food Expenditures}

The csv file \verb`aus_food.csv` contains records of total monthly expenditure on
cafes, restaurants and takeaway food services in Australia (\$billion) from
2004 through 2016. (Hyndman, R.J., \& Athanasopoulos, G. (2018)
Forecasting: principles and practice, 2nd edition,
OTexts: Melbourne, Australia. OTexts.com/fpp2. Accessed on July 22, 2019.)

The file contains two variables: \verb`expenditure` is the monthly expenditure
on food services, and \verb`time` encodes the year and month. For example,
January 2004 is coded as $2004$, and February 2004 is encoded as $2004.0833$...
since $1/12$ is about $0.083$.
In this example we will build a model to forecast monthly food expenditures.
We will fit the model to the data up through 2015, and predict food expenditures
in 2016.

1. Read the data in.  Make a plot that has food expenditures on the vertical
axis and time on the horizontal axis.


2. Your plot in part 1 showed an increasing trend in food expenditures over
time, with fairly consistent seasonal patterns (for instance, there is a
regular peak in food expenditures in December of each year).  Let's use a
model for food expenditures that has a linear function of time to capture
the long-term increasing trend, and sine and cosine terms to capture
seasonality around that trend. Specifically, if we use $y_t$ to denote food
expenditures at time $t$, we have:

\begin{equation*}
y_t \approx \alpha_0 + \alpha_1 t + \sum_{k = 1}^K \{\gamma_{k} \sin\left( 2 \pi k t \right) + \gamma^*_{k} \cos\left( 2 \pi k t \right) \}
\end{equation*}

We will use $K = 6$ trigonometric terms.  We will explore
using sums of trigonometric terms like this more later in the course when we discuss
Fourier series.  For now, let's just note that the first terms in this sum have $k = 1$,
which means those terms have period 1 year, which is reasonable for capturing
seasonality with a period of 1 year.

We can express this model in matrix form as $y \approx X \beta$, where for the data from
2004 through 2015, we have:

\begin{align*}
y &= \begin{bmatrix} y_1 \\ \vdots \\ y_t \end{bmatrix} \\
X &= \begin{bmatrix}
1 & 2004 & \sin\left( 2 \pi \cdot 1 \cdot 2004 \right) & \cos\left( 2 \pi \cdot 1 \cdot 2004 \right) & \cdots & \sin\left( 2 \pi \cdot 6 \cdot 2004 \right) & \cos\left( 2 \pi \cdot 6 \cdot 2004 \right) \\
1 & 2004.083 & \sin\left( 2 \pi \cdot 1 \cdot 2004.083 \right) & \cos\left( 2 \pi \cdot 1 \cdot 2004.083 \right) & \cdots & \sin\left( 2 \pi \cdot 6 \cdot 2004.083 \right) & \cos\left( 2 \pi \cdot 6 \cdot 2004.083 \right) \\
\vdots & \vdots & \vdots & \vdots & \ddots & \vdots & \vdots \\
1 & 2015.917 & \sin\left( 2 \pi \cdot 1 \cdot 2015.917 \right) & \cos\left( 2 \pi \cdot 1 \cdot 2015.917 \right) & \cdots & \sin\left( 2 \pi \cdot 6 \cdot 2015.917 \right) & \cos\left( 2 \pi \cdot 6 \cdot 2015.917 \right) \\
\end{bmatrix} \\
\beta &= \begin{bmatrix} \alpha_0 \\ \alpha_1 \\ \gamma_1 \\ \gamma^*_1 \\ \vdots \\ \gamma_6 \\ \gamma^*_6 \end{bmatrix}
\end{align*}


In python, create the matrix $X$ using the data from 2004 up through 2015.  For each column of X, create a plot showing the values in that column on the vertical axis and the corresponding values of t on the horizontal axis.  Do your results make sense?

3. Find the least squares estimate of $\beta$.

4. Create a matrix $X_{new}$ corresponding to the data for 2016.

5. By combining your estimate of $\beta$ from part 3 and your $X_{new}$ from part 4, obtain predicted values for food expenditures in every month of 2016.

6. Create a plot comparing your predictions of food expenditure to the actual observed food expenditures in 2016.

\end{document}
